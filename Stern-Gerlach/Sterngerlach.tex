%Layout: Kopf- und Fusszeilen
\documentclass[12pt,a4paper]{article} %Schriftgrösse
\usepackage[german,english,ngerman]{babel} %ansonsten "ngerman"
\selectlanguage{ngerman}

\usepackage[left=3cm,right=3cm,top=3cm,bottom=4cm]{geometry} \setlength{\headsep}{1.5cm}\usepackage{scrpage2}
\pagestyle{scrheadings}

\ihead{Praktikumsbericht, Physik III}\chead{} \ohead{Zeeman-Effekt} \cfoot{\thepage} %Kopfzeile,Fusszeile
\setheadsepline{0.2pt}\setheadtopline{0.2pt} 
\renewcommand{\baselinestretch}{1.25} %Zeilenabstand
\setlength{\parindent}{0pt} %verhindert Absatzeinzug
\setlength{\parskip}{0pt} %verhindert Absatzabstand
\usepackage[normalem]{ulem} %Unterstreichungen möglich
\usepackage{setspace}

%Sprachwahl (Umlaute)
\usepackage[T1]{fontenc} %Schriftencoding
\usepackage[utf8]{inputenc}%input encoding

\usepackage{graphicx} %Bilder einbinden %\graphicspath{{_Bilder}}
\usepackage{amsmath} %Gleichheitszeichen untereinander
\usepackage{amssymb}
\usepackage{amsfonts}
\usepackage{amsthm}
%\usepackage{here}
\usepackage{floatflt} %damit Text neben Bildern
\usepackage{mathtools}
\usepackage{mathptmx} %für Fonts
\usepackage{array}
\usepackage{pgf}
\usepackage{pifont}
\usepackage{tikz}
%\usepackage[pdftex]{hyperref}
%\hypersetup{bookmarks=false,colorlinks=true}
\usepackage{url}
\usepackage{enumitem} %damit Abstand bei Aufzählung veränderbar

\newcommand*{\defeq}{\mathrel{\vcenter{\baselineskip0.5ex \lineskiplimit0pt
                     \hbox{\scriptsize.}\hbox{\scriptsize.}}}% ":= schöner"
                     =}

%*********************************************************************
\begin{document}
%Titelseite
\begin{titlepage} \begin{figure}[h] \hfill \includegraphics[scale=0.04]{uzh.png} \end{figure}
\vspace{2 cm}
\textbf{\large{Praktikumsbericht, Physik III}} \\ \vspace{1 cm} \\ \textbf{\huge{Zeeman-Versuch}} \\ \normalsize \\ \vspace{1 cm}
\par
\begingroup
\leftskip 0 cm
\rightskip\leftskip
\textbf{Modul:}\\ PHY131 \\ \\
\textbf{Assistent:}\\ Ruth (@physik.uzh.ch)\\ \\
\textbf{Studierende:}\\
Laura Burri (lauradaniela.burri@uzh.ch)\\
Nora Salgo (nora.salgo@uzh.ch)\\
Fabian Stäger (fabian.staeger@uzh.ch) \\ \\
\textbf{Datum des Experiments:}\\ 18.01.2017 \\ \\
\par \endgroup \end{titlepage} \clearpage

%Einleitung
\newpage
\tableofcontents
\newpage
\section{Einleitung}
\subsection{Theoretische Vorüberlegungen}


\newpage
\section{Messergebnisse}

Werte von Fit für q
\begin{table}[h] \begin{center}
		\begin{tabular}{|lll|} \hline
			T [$^{\circ}$C] & I [mA] & q [mm]\\ \hline
			$185.9$ & $403$ & $1.8257 \pm 0.0103$\\
			$185.8$ & $500$ & $2.4297 \pm 0.0137$ \\
			$185.6$ & $700$ & $3.7338 \pm 0.0217$ \\
			$185.2$ & $800$ & $4.4131 \pm 0.0268$\\
			$185.0$ & $900$ & $5.8741 \pm 0.0343$\\
			$186.4$ & $1000$ & $5.6536 \pm 0.0349$ \\ \hline
	\end{tabular} \end{center}
\caption{q-Werte}
\end{table}




\newpage
\section{Anhang}

\newpage
\subsection{Fehlerrechnung}




%%%%%%%%%%%%
\end{document}

\begin{table}[h] \begin{center}
		\begin{tabular}{p{0.7cm}|ll}
		i & $x_{0i} \ [cm]$  \\ \hline
		1&  23.90 $\pm$ 0.05\\ 
		2&  23.95 $\pm$ 0.05\\ 
		3&  24.00 $\pm$ 0.05\\ 
		4&  23.95 $\pm$ 0.05\\   
		5&  24.00 $\pm$ 0.05\\
		\end{tabular} \end{center}
	\end{table}

\begin{center} \includegraphics[scale=0.14]{r-aufbau.jpeg} \end{center}

\begin{floatingfigure}[r]{5 cm} \begin{flushleft} \includegraphics[scale=0.08]{r-spiral.jpeg} \end{flushleft} \end{floatingfigure}

\begin{floatingfigure}[l]{8 cm} \begin{flushleft} \includegraphics[scale=0.17]{r-aufbau.jpeg} \end{flushleft} \end{floatingfigure}

